% Copyright 2009 by Joyan <joyanlj@gmail.com>.
% Last Modified Feb. 20, 2009
%
% In principle, this file can be redistributed and/or modified under
% the terms of the GNU Public License, version 2.
%
% However, this file is supposed to be a template to be modified for
% your own needs. For this reason, if you use this file as a template
% and not specifically distribute it as part of a another
% package/program, I grant the extra permission to freely copy and
% modify this file as you see fit and even to delete this copyright
% notice.

\documentclass[a4paper,dvipdfm]{article}
\usepackage[BoldFont,SlantFont,CJKnumber]{xeCJK}%配合xetex >0.997解决
                                %中英文显示问题
\usepackage{indentfirst}%用于首行缩进
\usepackage[width=500pt]{geometry}%用于设置页面大小
\usepackage[dvipdfm]{graphicx}
\usepackage[usenames,dvipsnames]{color}
\usepackage{xcolor}
%\usepackage{threeparttable}%用于表格内注释
%\usepackage[centerlast]{caption2} %浮动图形和表格标题样式
%\usepackage[rightcaption]{sidecap}%轻松的得到标题在一边的浮动图形或表格
%\usepackage{colortbl}%彩色表格
%\usepackage{multirow}%排版表格里的跨行文本数据和对齐方式
%\usepackage{textcomp}%常见数理单位和货币符号
%\usepackage[section]{placeins}%立即处理浮动体,section选项要求浮动体在它们所在的章节中排出
%\usepackage{subfigure} %%图形或表格并排排列

% \usepackage{fancyhdr}%设置页眉页脚
% \fancypagestyle{plain}{%
%   \fancyhf{}              
%   \fancyhead[R]{\thepage}
%   \fancyhead[L]{00648333}
% }
% \fancyhead[R]{\thepage}
% \fancyhead[L]{00648333}
% \pagestyle{fancy}

% \usepackage{setspace}%设置行距
% \doublespacing%双倍行距

% \usepackage[dvipdfm,a4paper,CJKbookmarks,bookmarks=true,bookmarksopen=true]{hyperref}%很强大的一个东西
% \hypersetup{
%     pdftitle={},
%     pdfauthor={Wen Zhang},
%     pdfkeywords={},
%     bookmarksnumbered,
%     pagebackref=true,
%     breaklinks=true,
% %    pdfview=FitH,       % Or try pdfstartview={FitV}, This lead to uncorrect bookmarks
%     urlcolor=cyan,
%     colorlinks=true,
%     citecolor=magenta,          %citeref's color
%     linkcolor=magenta,
%         }



\DeclareGraphicsExtensions{.eps,.ps,.eps.gz,.ps.gz,.eps.Z,.EPS,.pdf,.PDF}%声明图片后缀

\newcommand{\chuhao}{\fontsize{42pt}{\baselineskip}\selectfont}
\newcommand{\xiaochuhao}{\fontsize{36pt}{\baselineskip}\selectfont}
\newcommand{\yihao}{\fontsize{28pt}{\baselineskip}\selectfont}
\newcommand{\erhao}{\fontsize{21pt}{\baselineskip}\selectfont}
\newcommand{\xiaoerhao}{\fontsize{18pt}{\baselineskip}\selectfont}
\newcommand{\sanhao}{\fontsize{15.75pt}{\baselineskip}\selectfont}
\newcommand{\sihao}{\fontsize{14pt}{\baselineskip}\selectfont}
\newcommand{\xiaosihao}{\fontsize{12pt}{\baselineskip}\selectfont}
\newcommand{\wuhao}{\fontsize{10.5pt}{\baselineskip}\selectfont}
\newcommand{\xiaowuhao}{\fontsize{9pt}{\baselineskip}\selectfont}
\newcommand{\liuhao}{\fontsize{7.875pt}{\baselineskip}\selectfont}
\newcommand{\qihao}{\fontsize{5.25pt}{\baselineskip}\selectfont}

\defaultfontfeatures{Mapping=tex-text}
\setmainfont{Droid Sans}% 设置缺省英文字体
\setCJKmainfont[BoldFont=SimHei,ItalicFont=KaiTi_GB2312]{SimSun}% 设置缺省中文字体

\renewcommand\refname{参考文献}

\newcommand{\upcite}[1]{\textsuperscript{\cite{#1}}} %自定义命令\upcite, 使参考文献引用以上标出现

\graphicspath{{figures/}}

\renewcommand\contentsname{目录}
\renewcommand\listfigurename{插图索引}
\renewcommand\listtablename{表格索引}
\newcommand\listequationname{公式索引}
\newcommand\equationname{公式}
\renewcommand\refname{参考文献}
\renewcommand\indexname{索引}
\renewcommand\figurename{图}
\renewcommand\tablename{表}
% \newcommand\CJKprechaptername{第}
% \newcommand\CJKchaptername{章}
% \newcommand\CJKthechapter{\@arabic\c@chapter}
% \renewcommand\chaptername{\CJKprechaptername~\CJKthechapter~\CJKchaptername}
\renewcommand\appendixname{附录}

\makeatletter
\newenvironment{tablehere}
  {\def\@captype{table}}
  {}

\newenvironment{figurehere}
  {\def\@captype{figure}}
  {}
\makeatother



\begin{document}

%%%%%%%%%%%%%%%%%%%%%%%%%%% 正文开始 %%%%%%%%%%%%%%%%%%%%%%%%%%%%%%


\title{古罗马凯旋式和凯旋门}
\author{陈志杰 \\ zhijiechn@gmail.com}
\date{May. 2009}

\maketitle

\tableofcontents

\section{绪论(总括凯旋门)}

门,是人们在日常生活中再熟悉不过的一类建筑,一般来说,只要存在两个世界
的分隔,只要不是完全隔绝,总会有门的存在,门的存在,连接了两个不同世界。
基于门的以上用途,人们给门赋予了许多象征的意义,如门本身预示着一个新的
环境即将到来,代表新的开始和新的纪元;从门內经过,代表着一个世界或者事
件的终结,也预示着新的世界或者事件的开始\textbf{CITATION NEEDED};不仅
如此,从门内经过,还被世界各地的人们用来进行各种仪式,如中国古代的皇
帝,每次出城围猎,往往在城门周围聚集着各种各样的人,围猎的队伍也是在经
过城门的时候走的最``雄赳赳气昂昂'';其他的各种出发仪式和归来仪式也大都
是围绕着门来进行的。在所有这些门和有关门的仪式中,最为庄严雄伟的,当数
古罗马人的凯旋门和凯旋式。

\section{古罗马凯旋式}

说到古罗马凯旋门,就不得不先提一下古罗马的凯旋式。罗马凯旋式(The Roman
triumph, or triumphus)\textbf{C.N.},是古罗马文明的一项重要的与宗教有关
的仪式。关于这项仪式的起源,目前史学界还有许多争议,本文也暂不讨论,但
是其目的则是比较明确的:用来纪念某个帝王及其军队在重要战争中的胜利,也
同时纪念某一个战争的结束。这种纪念仪式

\textbf{TO BE CONTINUED,查阅中文wiki凯旋式,弄清楚凯旋门是在什么地方用
  到。}

\section{凯旋门总述}

\textbf{查阅中文wiki,将以下文字准确翻译成中文}.

A triumphal arch is a structure in the shape of a monumental archway,
in theory built to celebrate a victory in war, actually used to
celebrate a ruler.

Roman classical triumphal arch was a free-standing structure, quite
separate from city gates or walls, but the form is often used in
engaged arches as well. In its simplest form a triumphal arch consists
of two massive piers connected by an arch, crowned with a flat
superstructure or attic on which a statue might be mounted or which
bears commemorative inscriptions. The structure should be decorated
with carvings, notably including "Victories", winged female figures
(very similar to angels), a pair of which typically occupy the curved
triangles beside the top of the arch curve. More elaborate triumphal
arches have flanking subsidiary archways, typically a pair.

The rhythmic ABA motif—of central arched void flanked by smaller
ones—was adapted in Classical architecture, particularly since the
Renaissance, to articulate the walls of structures. The voids may take
the form of niches or be "blind", with masonry continuous behind.



\subsection{凯旋门的用途}

\subsection{凯旋门的结构}

\subsection{凯旋门的复用}

\section{古罗马凯旋门}

The tradition dates back to ancient Rome and is connected to the
Senate's custom of granting Roman triumphs. Surprisingly little is
known about how the Romans used triumphal arches; the only ancient
author who discussed them was Pliny the Elder, writing in the first
century AD. They are not mentioned at all by Vitruvius, the first
century BC writer on Roman architecture. Pliny describes them as being
honorary monuments of unusual importance, erected to commemorate
triumphs. By the second century arches were being erected to
commemorate other events, such as the surviving triumphal arch at
Ancona, erected by a grateful city to commemorate Trajan's
improvements to the harbor.

It is unclear when the Romans first began erecting triumphal arches.
They originated some time during the Roman Republican era, during
which time three were erected in Rome, the earliest being one to
Lucius Stertinius built in 196 BC. These appear to have been temporary
structures, and none now survive. Most triumphal arches were built
during the Roman Empire. By the fourth century, thirty-six triumphal
arches can be traced in Rome. Only five now survive (see list below).

The arches of Rome became increasingly elaborate over the centuries.
They were at first very simple symbolic temporary gateways to the
city, being built of brick or stone with a semicircular arched heading
and hung with trophies of captured arms. Later arches were built of
high-quality marble with a large central arch in the middle, its
ceiling treated as a barrel vault, and sometimes two smaller ones on
each side, adorned with a complete Architectural order, of columns and
entablature, enriched with symbolic or narrative bas-reliefs and
crowned with bronze statues, often a quadriga. The festive Corinthian
order was the usual one.

\subsection{古罗马是凯旋门的全盛时期}
By the fourth century, thirty-six triumphal arches can be traced in
Rome. Only five now survive (see list below).
\subsection{古罗马凯旋门的兴起}

\textbf{Arc d’Arcadius, Honorius et Théodose 405 BC}

\textbf{Arc de Scipion l'Africain 109 BC}

\subsection{奥古斯都凯旋门(Aosta) 35 BC}

Outside the town is a triumphal arch in honour of Augustus, built in
35 BC to celebrate the victory of consul Varro Murena over the
Salassi. About 8 km to the west is a single-arched Roman bridge,
called the Pont d'Aël. It has a closed passage, lighted by windows for
foot passengers in winter, and above it an open footpath, both being
about 10 m in width.

\subsection{奥古斯都凯旋门(Fano)}

\subsection{奥古斯都凯旋门(罗马) 29 BC}

\subsection{奥古斯都凯旋门(Rimini) 27 BC}

\subsection{Pont Flavien(Bridge) 12 BC}
\label{sec:pont-flav-12}

\subsection{Arch of Drusus 9 BC}
\label{sec:arch-drusus-9}

\textbf{Arc de Lentulus et Crispinus 2 AD}

\textbf{Arc de Dolabella et Silanus 10 AD}

\textbf{Arcs de Tibère 16 AD}

\subsection{Arch of Germanicus 18/19 AD}
\label{sec:arch-germanicus-1819}

这个就是Arcs de Drusus et Germanicus吧?两个门一样大?

\textbf{Arch of Drusus 23 AD}

\subsection{Triumphal Arch of Orange 27 AD}
\label{sec:triumph-arch-orange}

\subsection{Arch of Claudius 46-51 AD}

\textbf{Arc de Néron 58-62 AD}

\subsection{Arch of Titus 81 AD}

\subsection{Arch of Hadrian 131-132 AD}
\label{sec:arch-hadrian-131}

\textbf{Arc de Marc Aurèle 176 AD}

\subsection{Arch of Septimius Severus (Roman Forum) 203 AD}
\label{sec:arch-sept-sever}

\textbf{Arc de Septime Sévère (forum Boarium) 204 AD}

\subsection{Arch of Gallienus 262 AD}
\label{sec:arch-gallienus-262}

\subsection{Arch of Galerius 298-303 CE}
\label{sec:arch-galerius-298}

\subsection{Arch of Constantine 315 AD}

\textbf{Arc du divin Constantin Arc de Janus quadrifrons 356 AD}

\textbf{Arc de Gratien, Valentinien et Théodose 379-383 AD}


\subsection{特洛伊拱门 114-115}

\subsection{Golden Gate ???}
\label{sec:golden-gate-}

\subsection{The triumphal arch of Glanum ???}
\label{sec:triumph-arch-glan}



\subsection{古罗马中期凯旋门}

\subsection{古罗马凯旋门的顶峰}

\section{后凯旋门时代}

Triumphal arches in the Roman style were revived during the
Renaissance, when there was a Europe-wide upswelling of interest in
the art and architecture of ancient Rome. Between the 15th and 19th
century, kings and emperors erected numerous triumphal arches in
conscious imitation of the Roman tradition. One of the earliest was
the "Aragonese Arch" at the Castel Nuovo in Naples, erected by Alfonso
V in 1443, although like the later Porta Capuana this was engaged as
part of the entrance to the castle. Temporary examples were erected in
enormous numbers for festivities such as Royal Entries from the late
Middle Ages onwards. The Emperor Maximilian I commissioned the artist
Albrecht Dürer to design an elaborately decorated monumental arch in
woodcut for him (3.75 metres high, in 192 different sheets), which was
never intended to be built, but was printed in an edition of 700
copies and distributed to be coloured and pasted on the walls of large
rooms. Louis XIV of France and Napoleon Bonaparte both erected arches
to commemorate their military triumphs, most famously the Arc de
Triomphe in Paris. Arches were erected for similar purposes in the
U.K., the United States, Germany, Romania, Russia and Spain, amongst
other countries. Built to honour and glorify President Kim Il Sung and
modeled after the Arc de Triomphe in Paris, the Arch of Triumph in
Pyongyang is the largest triumphal arch in the world (although the
Grande Arche at La Défense near Paris is much larger, it is not a
triumphal arch). A far larger arch was planned for Berlin by Adolf
Hitler and his architect Albert Speer, but construction was never
begun.

Temporary triumphal arches are still constructed, intended to be used
for a celebratory parade or ceremony and then be dismantled
afterwards.

The term triumphal arch is also often used of the arch separating the
nave from the apse of a church in basilica form, often decorated with
mosaics or paintings.



\section{总结:凯旋门的价值及其对现代建筑的影响}

\appendix{}

\bibliography{bibdb}
\bibliographystyle{unsrt}


\end{document}
%%% Local Variables: 
%%% mode: latex
%%% TeX-master: t
%%% End: 
