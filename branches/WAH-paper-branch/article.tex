% Copyright 2009 by Joyan <joyanlj@gmail.com>.
% Last Modified Feb. 20, 2009
%
% In principle, this file can be redistributed and/or modified under
% the terms of the GNU Public License, version 2.
%
% However, this file is supposed to be a template to be modified for
% your own needs. For this reason, if you use this file as a template
% and not specifically distribute it as part of a another
% package/program, I grant the extra permission to freely copy and
% modify this file as you see fit and even to delete this copyright
% notice.

\documentclass[a4paper,dvipdfm]{article}
\usepackage[BoldFont,SlantFont,CJKnumber]{xeCJK}%配合xetex >0.997解决
                                %中英文显示问题
\usepackage{indentfirst}%用于首行缩进
\usepackage[width=500pt]{geometry}%用于设置页面大小
\usepackage[dvipdfm]{graphicx}
\usepackage[usenames,dvipsnames]{color}
\usepackage{xcolor}
%\usepackage{threeparttable}%用于表格内注释
%\usepackage[centerlast]{caption2} %浮动图形和表格标题样式
%\usepackage[rightcaption]{sidecap}%轻松的得到标题在一边的浮动图形或表格
%\usepackage{colortbl}%彩色表格
%\usepackage{multirow}%排版表格里的跨行文本数据和对齐方式
%\usepackage{textcomp}%常见数理单位和货币符号
%\usepackage[section]{placeins}%立即处理浮动体,section选项要求浮动体在它们所在的章节中排出
%\usepackage{subfigure} %%图形或表格并排排列

% \usepackage{fancyhdr}%设置页眉页脚
% \fancypagestyle{plain}{%
%   \fancyhf{}              
%   \fancyhead[R]{\thepage}
%   \fancyhead[L]{00648333}
% }
% \fancyhead[R]{\thepage}
% \fancyhead[L]{00648333}
% \pagestyle{fancy}

% \usepackage{setspace}%设置行距
% \doublespacing%双倍行距

% \usepackage[dvipdfm,a4paper,CJKbookmarks,bookmarks=true,bookmarksopen=true]{hyperref}%很强大的一个东西
% \hypersetup{
%     pdftitle={},
%     pdfauthor={Wen Zhang},
%     pdfkeywords={},
%     bookmarksnumbered,
%     pagebackref=true,
%     breaklinks=true,
% %    pdfview=FitH,       % Or try pdfstartview={FitV}, This lead to uncorrect bookmarks
%     urlcolor=cyan,
%     colorlinks=true,
%     citecolor=magenta,          %citeref's color
%     linkcolor=magenta,
%         }



\DeclareGraphicsExtensions{.eps,.ps,.eps.gz,.ps.gz,.eps.Z,.EPS,.pdf,.PDF}%声明图片后缀

\newcommand{\chuhao}{\fontsize{42pt}{\baselineskip}\selectfont}
\newcommand{\xiaochuhao}{\fontsize{36pt}{\baselineskip}\selectfont}
\newcommand{\yihao}{\fontsize{28pt}{\baselineskip}\selectfont}
\newcommand{\erhao}{\fontsize{21pt}{\baselineskip}\selectfont}
\newcommand{\xiaoerhao}{\fontsize{18pt}{\baselineskip}\selectfont}
\newcommand{\sanhao}{\fontsize{15.75pt}{\baselineskip}\selectfont}
\newcommand{\sihao}{\fontsize{14pt}{\baselineskip}\selectfont}
\newcommand{\xiaosihao}{\fontsize{12pt}{\baselineskip}\selectfont}
\newcommand{\wuhao}{\fontsize{10.5pt}{\baselineskip}\selectfont}
\newcommand{\xiaowuhao}{\fontsize{9pt}{\baselineskip}\selectfont}
\newcommand{\liuhao}{\fontsize{7.875pt}{\baselineskip}\selectfont}
\newcommand{\qihao}{\fontsize{5.25pt}{\baselineskip}\selectfont}

\defaultfontfeatures{Mapping=tex-text}
\setmainfont{Droid Sans}% 设置缺省英文字体
\setCJKmainfont[BoldFont=SimHei,ItalicFont=KaiTi_GB2312]{SimSun}% 设置缺省中文字体

\renewcommand\refname{参考文献}

\newcommand{\upcite}[1]{\textsuperscript{\cite{#1}}} %自定义命令\upcite, 使参考文献引用以上标出现

\graphicspath{{figures/}}

\renewcommand\contentsname{目录}
\renewcommand\listfigurename{插图索引}
\renewcommand\listtablename{表格索引}
\newcommand\listequationname{公式索引}
\newcommand\equationname{公式}
\renewcommand\refname{参考文献}
\renewcommand\indexname{索引}
\renewcommand\figurename{图}
\renewcommand\tablename{表}
% \newcommand\CJKprechaptername{第}
% \newcommand\CJKchaptername{章}
% \newcommand\CJKthechapter{\@arabic\c@chapter}
% \renewcommand\chaptername{\CJKprechaptername~\CJKthechapter~\CJKchaptername}
\renewcommand\appendixname{附录}

\makeatletter
\newenvironment{tablehere}
  {\def\@captype{table}}
  {}

\newenvironment{figurehere}
  {\def\@captype{figure}}
  {}
\makeatother



\begin{document}

%%%%%%%%%%%%%%%%%%%%%%%%%%% 正文开始 %%%%%%%%%%%%%%%%%%%%%%%%%%%%%%


\title{古罗马凯旋式和凯旋门}
\author{陈志杰 \\ zhijiechn@gmail.com}
\date{May. 2009}

\maketitle

\tableofcontents

\section{绪论(总括凯旋门)}

门,是人们在日常生活中再熟悉不过的一类建筑,一般来说,只要存在两个世界
的分隔,只要不是完全隔绝,总会有门的存在,门的存在,连接了两个不同世界。
基于门的以上用途,人们给门赋予了许多象征的意义,如门本身预示着一个新的
环境即将到来,代表新的开始和新的纪元;从门內经过,代表着一个世界或者事
件的终结,也预示着新的世界或者事件的开始\textbf{CITATION NEEDED};不仅
如此,从门内经过,还被世界各地的人们用来进行各种仪式,如中国古代的皇
帝,每次出城围猎,往往在城门周围聚集着各种各样的人,围猎的队伍也是在经
过城门的时候走的最``雄赳赳气昂昂'';其他的各种出发仪式和归来仪式也大都
是围绕着门来进行的。在所有这些门和有关门的仪式中,最为庄严雄伟的,当数
古罗马人的凯旋门和凯旋式。

\section{古罗马凯旋式}

说到古罗马凯旋门,就不得不先提一下古罗马的凯旋式。罗马凯旋式(The Roman
triumph, or triumphus)\textbf{C.N.},是古罗马授予取得重大军事成果,特别
是那些获得打赢了一整场战争的军事将领的庆祝仪式。对于统治罗马的贵族而
言,凯旋式是最大且最受欢迎的荣耀。获得凯旋式的将领被称为凯旋者(或凯旋
英雄,拉丁文:vir triumphalis),并且有权在其余生中保留使用这一称号的权
利。在他死后,每当其家族举行葬礼时,都会雇佣一名演员戴上他的死亡面
具(拉丁文:imago),穿上其在凯旋式上穿着过的紫色绣金的刺绣托加相同的托
加,以彰显其生前的成就。关于这项仪式的起源,目前史学界还有许多争议,本
文也暂不讨论,但是其目的则是比较明确的:用来纪念某个军事将领及其军队在
重要战争中的胜利,也同时纪念某一个战争的结束。这种纪念仪式的兴起早于罗
马建国,但在东罗马帝国时期达到顶峰。\textbf{C.N}

凯旋式的最主要环节是从在瑟维尔墙(\textbf{英文})之外的战神广
场(\textbf{应为})上一直到卡皮托林山(\textbf{英文})上的神庙中的一长
段庄严隆重的环节,由于本文主要讨论的内容古罗马凯旋门,因而这一部分在此
不展开详述,详见\textbf{C.N.}。

在凯旋式上,通常还有新的纪念物被树立以纪念着这场凯旋式,其中最壮观的,
要数各种各样的凯旋门。凯旋门与一般的城门不同,通常为横跨在一条道路之上
的独立性建筑,在凯旋门下经过,也往往更多是一种仪式和象征。


\section{古罗马凯旋门}

\subsection{凯旋门的主要结构}

凯旋门是一个巨大的拱形建筑,典型的古罗马凯旋门是独立于城墙的,也就是多
它并不具有一般门的那种为分割的区域提供捷径的功能,而更多的是作为一种符
号和纪念引起人们的注意。他以两个柱子以及一个横跨在柱子之上的一个拱形作
为主要骨架,上着各种各样的装饰性花纹和图案。由于凯旋门巨大的结构和对于
稳定性的严格要求,因此主要骨架并不会被精雕细琢,所有的装饰都是以附在主
干上的各种各样栩栩如生的浮雕为主,这些浮雕的主要内容大多为歌颂胜利为
主,浮雕上的人物主要有建筑所纪念的对象和事件,军队和战利品,带有翅膀的
女性形象(与后来的宗教作品中的天使比较类似),拱形曲线等。当然,在门的
正上方,一般还会刻有与这次凯旋式或者胜利有关的题词,这也是获取某座凯旋
门的纪念对象和背景的最重要的依据。有些更加精致的凯旋门还会在大门的两测
开两个小拱门(附属拱门)。

\subsection{凯旋门的历史}

虽然凯旋门早在罗马建国以前就已经出现,但是凯旋门真正发展成熟,直至成为
一件令人惊叹的艺术品,还是要归功于古罗马帝国帝王对于这种纪念形式的热衷。
在初期,凯旋门只是用砖和石头建造的非常简单的临时性纪念建筑,拱上装饰有
奖杯和各种战利品;随着罗马帝国国立不断强盛和建造技术的不断发展,凯旋门
的用料和制作工艺页越来越讲究,最终发展成为用上等大理石建造的,中间承放
大型拱石,并被精雕细琢的艺术品。当然,建造凯旋门所用的时间以及花费的人
力物力也随其审美价值和雄伟程度的提高而不断增加。

值得注意的是,有很多的凯旋门上的图案,都是从倍的凯旋门上挪移过来的,如
君士坦丁凯旋门(\textbf{英文})这一罗马史上最大的拱门,上面的浮雕就是采
自许多以前建造的凯旋门,使得着座建筑上的装饰风格各异,``兼容并包''。这
样的情况在这种纪念性的建筑物的建造中并不多见,究竟是出于节省开销的目
的,还是另有原因?这也是目前大家争论的话题之一\textbf{C.N.}。

\section{古罗马凯旋门纵览}

从罗马建国到公元5世纪,历史上记载的资料显示罗马共建造了三十四座凯旋门
式的拱形建筑,其中大部分现在已经见不到,只有一少部分至今依然存在,下面
我们就以时间和在位执政者的顺序,走进凯旋门的世界,了解和欣赏这一个个人
类智慧和力量的杰作。

\subsection{古罗马凯旋门的兴起}

这一时期的凯旋门并没有统一的制作流程和先进的工艺,大多是用砖石建造的临
时性建筑,因此,很遗憾,它们并没有经受住时间的冲刷,至今已经不存在与这
个世界上,只是从一些史料上能够找到他们的影子,这一时期有记载的凯旋门主
要包括阿卡尔狄乌斯拱门(\textsf{Latin:Arcus Arcadii Honorii et
  Theodosii, 405 BC})和非洲西皮翁拱门(\textsf{Arcus Scipii Africani,
  109 BC})

\subsection{奥古斯都大帝时期}

由于奥古斯都大帝在世界各地的赫赫战功,在他在位时期共建造了四座凯旋门,
分别在奥斯塔(Aosta)、法诺(Fano)、罗马和里米尼(Rimini)。

奥斯塔奥古斯都凯旋门(Arch of Augustus in Aosta, 35
BC)\textbf{PIC},建于公元前35年,是为了庆祝奥古斯都大帝在 ``consul
Varro Murena over the Salassi'' 战争中的胜利而建造的,从图中我们可以看
出,此时的建筑仍然是以砖石为主,结构比较简单。

罗马的奥古斯都凯旋门是古罗马广场上的一座凯旋门,完成于公元前29年,庆祝
奥古斯都在亚克兴角战役战胜马克·安东尼和克利奥帕特拉七世。奥古斯都凯旋门
跨越卡斯托尔和波吕克斯神庙和凯撒神庙之间的道路,靠近灶神庙。1546年人们
在这一地点发现了献给奥古斯都的巨大碑刻,因此确定了奥古斯都凯旋门的存在。
奥古斯都凯旋门本身留下的遗迹很少,但是它出现在当时的货币上。它有三个通
道,在罗马首次出现这种样式的拱门,并为塞维鲁凯旋门和后来的君士坦丁凯旋
门所仿效。图\textbf{PIC}是当时刻有凯旋门图像的钱币,图\textbf{PIC}是加
州大学洛杉矶分校的``Virtual Roman Forum''项目发布的罗马奥古斯都凯旋门的
复原像。从这些资料中我们可以看出这座凯旋门是三门建筑,支撑柱和上拱比较
细,缺乏后来的凯旋门那种庄重威严的气势,不错由此也可以看出罗马前后期建
筑风格从实用主义到越来越花哨的转变。

里米尼的奥古斯都凯旋门是这四个中保存最完整的一个,它建于公元前27年,是
为了纪念奥古斯都建造的,这座凯旋门并不是独立建筑,而是作为城墙的一部分
而建造的,见图\textbf{PIC}。

位于法诺的奥古斯都凯旋门建于公元2年,从图中可以看出这座拱门并不是独立
建筑,同样是由于这一时期的凯旋门的建造并没有特别正式地作为一个单独的项
目和主题受到足够的重视,如图\textbf{PIC}。


\subsection{石结构的凯旋门的兴起}

随着古罗马建造技术的不断发展,建造凯旋门的材料逐渐由原来的砖结构转变成
了更加坚固的石结构,这一时期的凯旋门的特点是:

\begin{itemize}
\item 比起以前的凯旋门,规模和坚固程度有了一定程度的提升;
\item 门上的装饰性的结构开始增多,出现了小规模的浮雕;
\item 但是装饰的数量有限,门的主体骨架仍然占用料的主要部分;
\item 门的独立性有了一定程度的提升,但是仍然主要是与其他的建筑配合(特
  别是与桥),作为其他建筑的装饰。
\end{itemize}

这一时期有记载的凯旋门主要有庞特桥上的凯旋门(Arch on Pont Flavien, 12
BC)、德鲁苏斯凯旋门(Arch of Drusus, 三座,分别建于 9 BC, 19 AD 和23
AD)、兰图鲁斯凯旋门(Latin:Arcus Lentuli et Crispini, 2 AD)和多拉贝拉
凯旋门(Latin: Arcus Dolabellae et Silani, 10 AD)。

建于公元前9年的德鲁苏斯拱门,其样子更加像是古罗马初期的拱门的样
式(如图\textbf{PIC}),它连同建于公元23年建造的德鲁苏斯拱门都是比较简
单的小型拱门,而在公元18-19年,为纪念德鲁苏斯-日耳曼尼库斯(Drusus
Germanicus,因而这个拱门页常被称为是日耳曼尼库斯拱门)而建造的拱门(如
图\textbf{PIC}),连同上面提到的庞特(Pont Flavien)桥上的凯旋
门(如图\textbf{PIC}),则相对来说要雄伟的多,至少这两座拱门已经是方方
正正,有了后来的凯旋门的样子,从图中可以看出,拱顶开始有了一圈记述功绩
的浮雕。

\subsection{古罗马凯旋门的鼎盛时期}

这段时期是古罗马凯旋门建造技术高度发展,各种样式和浮雕层出不穷的时代,
由于古罗马帝国国立的逐渐强盛,以及劳动力和物产的逐渐丰富,出现了用名贵
的大理石建造的凯旋门,不仅如此,作为凯旋式中越来越重要的一部分,凯旋门
的纪念意义越来越受到人们特别是贵族的青睐,于是,凯旋门作为一种单独的艺
术建筑和象征符号,逐渐摒弃了原来与城墙、桥梁一起的功能性,而变成了一种
纯粹的独立的建筑存在起来。

这段时期,可以继续细分为前期、中期和后期三个阶段:

\subsubsection{前期:凯旋门的独立和定型}

正如这一节的小标题所暗示的那样,这一时期的凯旋门的主要特征是:

\begin{itemize}
\item 凯旋门作为一种重要的纪念和象征,逐渐从整体建筑中独立出来,单独作
  为一种独立的建筑艺术形式。
\item 凯旋门的结构逐渐成熟,样式逐渐固定下来,成为后来的凯旋门效仿的对
  象。
\item 凯旋门的装饰性元素逐渐增多,特别是记叙性和歌颂性的浮雕的出现和增
  多,成为欣赏凯旋门的一个重要方面,也为后来的史学研究提供了大量史料。
\end{itemize}

这一时期有记载的凯旋门主要有:奥朗日凯旋门(Triumphal Arch of Orange,
27 AD)、克劳迪乌斯凯旋门(Arch of Claudius, 46-51 AD)、提图斯凯旋
门(Arch of Titus,81 AD)和哈德良凯旋门(Arch of Hadrian,131-132
AD)。


\textbf{奥郎日凯旋门}

奥郎日凯旋门位于今法国奥郎日,关于其始建时间,其实目前仍然是一个争论的
话题,比较令人信服的一个说法是最初是为了纪念发生于公元前58至52年的高卢
战争以及奥古斯都第二军团,后来提贝里乌斯(Tiberius)皇帝为了纪念发生在
莱茵兰(Rhineland)的是日耳曼尼库斯对抗日耳曼军队的胜利重建了
它,从图\textbf{PIC}中我们可以看到,奥郎日凯旋门已经对拱上的装饰物和浮
雕进行了精雕细琢的强调,拱上方的部分几乎占整个门的一半,上方刻有海战的
场景、战利品以及罗马人与高卢和日耳曼人对抗的其他场景。

整个凯旋门开有三个通道,中间一个较大的,两边一边一个较小的,整个凯旋门
由四列长柱支撑。奥郎日凯旋门是现存最早的用这种结构建造的凯旋门,后来的
塞维勒斯凯旋门(Arch of Septimius Severus)以及著名的君士坦丁凯旋门都
是仿照这种结构建造的。

\textbf{克劳迪乌斯凯旋门}

克劳迪乌斯凯旋门是为了纪念克劳迪乌斯成功入侵不列颠而建造的,现在已经被
毁,仅存一些门上的题词和装饰,保存在卡比托里山博物馆(Capitoline
Museum)。

\textbf{提图斯凯旋门}

提图斯凯旋门(图\textbf{PIC})是意大利罗马市古罗马广场东南圣道上的一座
大理石单拱凯旋门,由图密善皇帝兴建于兄长提图斯去世后不久,纪念在公
元70年征服和摧毁耶路撒冷,终止66年开始的犹太人大起义。提图斯凯旋门建于
公园81年,是这一时期凯旋门中最具艺术魅力的一个。凯旋门几乎各处都刻有精
美的浮雕作品,尤其是拱门内壁上,拥有现存唯一的对耶路撒冷圣殿器物的描
绘,上面清楚描绘了灯台和小号,可能还有陈设饼桌子(如图\textbf{PIC})。
通道的顶部描绘了以被神化的提图斯为中心的方格浮雕,如图\textbf{PIC}。门
的正上方使用大写罗马字母提写的题词,写着``参议院和罗马人民(将此献给)
的神圣的提图斯奥古斯都,神圣的维斯帕先之子''。

提图斯凯旋门的单拱结构和方形的样式,被后来的许多凯旋门所模仿,如法国的
巴黎凯旋门。

\textbf{哈德良凯旋门}

建于131-132年的哈德良凯旋门是为了纪念罗马皇帝哈德良的到来和他对希腊雅
典城的贡献建造的,从外观上看,它与上述几个凯旋门都不相
同(\textbf{如图PIC})。哈德良凯旋门矗立在通往宙斯神殿的路上,由18公里
外运来的潘泰列克大理石建造。


\subsubsection{中期:罗马帝国最盛时期的标志}

这个时期,由于罗马帝国的国力在此达到了顶峰,是各种凯旋门遍地开花的时期。
这个时期有记载的凯旋门包括:塞维勒斯凯旋门(Arch of Septimius Severus,
203 AD)、加列奴拱门(Arch of Gallienus 262 AD)、伽列里乌斯凯旋
门(Arch of Galerius 298-303 CE)以及君士坦丁凯旋门(Arch of
Constantine 315 AD)等等。

\textbf{塞维勒斯凯旋门(或称塞维鲁凯旋门)}

塞维鲁凯旋门(Arch of Septimius Severus)是古罗马广场西北端的一座白色大理
石建筑,建于公元203年,以庆祝塞普蒂米乌斯·塞维鲁皇帝和他的两个儿子卡拉
卡拉和塞普提米乌斯·盖塔在194/195年和197-199年两次战胜波斯。

从图\textbf{PIC}中可以看到,塞维鲁凯旋门与提图斯拱门有很多相似之处,特
别是拱顶的方形平台和上面的题字,以及拱门内部通道上的浮雕和拱顶的方形图
案。后来的君士坦丁凯旋门也是如此。

除了位于古罗马广场的这座,据记载在罗马帝国博亚里昂市场附近还有一个用来
纪念赛维勒斯的凯旋门,建于公元204年,不过现已无从寻找。


\textbf{君士坦丁凯旋门}

最终,我们终于来到了古罗马现存最大最雄伟的凯旋门——君士坦丁凯旋门面
前。君士坦丁凯旋门建于公元312年,是罗马城现存的三座凯旋门中年代最晚
的一座。它是为庆祝君士坦丁大帝于公元312年彻底战胜他的强敌马克森提,并统
一帝国而建的。这是一座三个拱门的凯旋门,高21米,面阔25.7米,进深7.4米。
由于它调整了高与阔的比例,横跨在道路中央,显得形体巨大。凯旋门的里里外
外充满了各种浮雕,表面上看去,巨大的凯旋门和丰富的浮雕虽然气派很大,但
缺乏整体观念。原因是凯旋门的各个部分并非作为一个统一体而创作的,甚至其
中的大部分构件是从过去的一些纪念性建筑,如图拉真广场建筑上的横饰带、哈
德良广场上一系列盾形浮雕以及马克·奥尔略皇帝纪念碑上的八块镶板,拆除过来
的。尽管如此,它仍不失为一座宏伟壮观的凯旋门,尤其是它上面所保存的罗马
帝国各个重要时期的雕刻,是一部生动的罗马雕刻史。  君士坦丁经常被认为
是第一位皈依基督教的皇帝,但是在这座在315年为他而建的拱门上找不到任何基
督教的形象。他的洗礼仪式仅是在22年后在他临终之际举行的。尽管君士坦丁拱
门建造得格外雄伟气派,它却没有自己的独特风格。  这座拱门上面的绝大多
数装饰品其实都是取自于以前的皇帝们所建造的各种建筑物,这也表明罗马的艺
术已经开始走下坡路了。所以说,处在拱门上端(顶阁)两侧的那8座矩形浮
雕(rectangular rellefs)原先是一座纪念马库斯奥里列阿斯(161—180)的建
筑物上的装饰,只是这位皇帝的头像被重新雕刻成了君士坦丁的样子(还有3座浮
雕和它们属于同一系列,现存卡匹托尔博物馆。它们清楚地显示着马库斯·奥里列
阿斯皇帝的样子)。


\subsubsection{后期:古罗马帝国和凯旋门的安魂曲}



\subsection{特洛伊拱门 114-115}

\subsection{Golden Gate ???}
\label{sec:golden-gate-}

\subsection{The triumphal arch of Glanum ???}
\label{sec:triumph-arch-glan}



\subsection{后凯旋门时期}

Triumphal arches in the Roman style were revived during the
Renaissance, when there was a Europe-wide upswelling of interest in
the art and architecture of ancient Rome. Between the 15th and 19th
century, kings and emperors erected numerous triumphal arches in
conscious imitation of the Roman tradition. One of the earliest was
the "Aragonese Arch" at the Castel Nuovo in Naples, erected by Alfonso
V in 1443, although like the later Porta Capuana this was engaged as
part of the entrance to the castle. Temporary examples were erected in
enormous numbers for festivities such as Royal Entries from the late
Middle Ages onwards. The Emperor Maximilian I commissioned the artist
Albrecht Dürer to design an elaborately decorated monumental arch in
woodcut for him (3.75 metres high, in 192 different sheets), which was
never intended to be built, but was printed in an edition of 700
copies and distributed to be coloured and pasted on the walls of large
rooms. Louis XIV of France and Napoleon Bonaparte both erected arches
to commemorate their military triumphs, most famously the Arc de
Triomphe in Paris. Arches were erected for similar purposes in the
U.K., the United States, Germany, Romania, Russia and Spain, amongst
other countries. Built to honour and glorify President Kim Il Sung and
modeled after the Arc de Triomphe in Paris, the Arch of Triumph in
Pyongyang is the largest triumphal arch in the world (although the
Grande Arche at La Défense near Paris is much larger, it is not a
triumphal arch). A far larger arch was planned for Berlin by Adolf
Hitler and his architect Albert Speer, but construction was never
begun.

Temporary triumphal arches are still constructed, intended to be used
for a celebratory parade or ceremony and then be dismantled
afterwards.

The term triumphal arch is also often used of the arch separating the
nave from the apse of a church in basilica form, often decorated with
mosaics or paintings.



\section{总结:凯旋门的价值及其对现代建筑的影响}

\appendix{}
《世界文化》2007年11月刊,张献军 孙霞著 胜利与光荣:罗马凯旋式

苏维托尼乌斯(1996).“第八卷 神圣的韦帕芗传、神圣的提图斯传、图密善传
第12节”,罗马十二帝王传,张竹明 王乃新 蒋平译,商务印书馆,第310页.ISBN
7100022029.

\subsection{凯旋门的结构}

\subsection{凯旋门的复用}


\bibliography{bibdb}
\bibliographystyle{unsrt}


\end{document}
%%% Local Variables: 
%%% mode: latex
%%% TeX-master: t
%%% End: 
