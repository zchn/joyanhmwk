% Copyright 2009 by Joyan <joyanlj@gmail.com>.
% Last Modified Feb. 20, 2009
%
% In principle, this file can be redistributed and/or modified under
% the terms of the GNU Public License, version 2.
%
% However, this file is supposed to be a template to be modified for
% your own needs. For this reason, if you use this file as a template
% and not specifically distribute it as part of a another
% package/program, I grant the extra permission to freely copy and
% modify this file as you see fit and even to delete this copyright
% notice.

\documentclass[a4paper,dvipdfm]{article}
\usepackage[BoldFont,SlantFont,CJKnumber]{xeCJK}%配合xetex >0.997解决
                                %中英文显示问题
\usepackage{indentfirst}%用于首行缩进
\usepackage[width=500pt]{geometry}%用于设置页面大小
\usepackage{listings}%用于提供lstlisting代码显示环境
\usepackage[dvipdfm]{graphicx}
\usepackage[usenames,dvipsnames]{color}
\usepackage{xcolor}
%\usepackage{threeparttable}%用于表格内注释
%\usepackage[centerlast]{caption2} %浮动图形和表格标题样式
%\usepackage[rightcaption]{sidecap}%轻松的得到标题在一边的浮动图形或表格
%\usepackage{colortbl}%彩色表格
%\usepackage{multirow}%排版表格里的跨行文本数据和对齐方式
%\usepackage{textcomp}%常见数理单位和货币符号
%\usepackage[section]{placeins}%立即处理浮动体,section选项要求浮动体在它们所在的章节中排出
%\usepackage{subfigure} %%图形或表格并排排列

% \usepackage{fancyhdr}%设置页眉页脚
% \fancypagestyle{plain}{%
%   \fancyhf{}              
%   \fancyhead[R]{\thepage}
%   \fancyhead[L]{00648333}
% }
% \fancyhead[R]{\thepage}
% \fancyhead[L]{00648333}
% \pagestyle{fancy}

% \usepackage{setspace}%设置行距
% \doublespacing%双倍行距

% \usepackage[dvipdfm,a4paper,CJKbookmarks,bookmarks=true,bookmarksopen=true]{hyperref}%很强大的一个东西
% \hypersetup{
%     pdftitle={},
%     pdfauthor={Wen Zhang},
%     pdfkeywords={},
%     bookmarksnumbered,
%     pagebackref=true,
%     breaklinks=true,
% %    pdfview=FitH,       % Or try pdfstartview={FitV}, This lead to uncorrect bookmarks
%     urlcolor=cyan,
%     colorlinks=true,
%     citecolor=magenta,          %citeref's color
%     linkcolor=magenta,
%         }



\DeclareGraphicsExtensions{.eps,.ps,.eps.gz,.ps.gz,.eps.Z,.EPS,.pdf,.PDF}%声明图片后缀

\newcommand{\chuhao}{\fontsize{42pt}{\baselineskip}\selectfont}
\newcommand{\xiaochuhao}{\fontsize{36pt}{\baselineskip}\selectfont}
\newcommand{\yihao}{\fontsize{28pt}{\baselineskip}\selectfont}
\newcommand{\erhao}{\fontsize{21pt}{\baselineskip}\selectfont}
\newcommand{\xiaoerhao}{\fontsize{18pt}{\baselineskip}\selectfont}
\newcommand{\sanhao}{\fontsize{15.75pt}{\baselineskip}\selectfont}
\newcommand{\sihao}{\fontsize{14pt}{\baselineskip}\selectfont}
\newcommand{\xiaosihao}{\fontsize{12pt}{\baselineskip}\selectfont}
\newcommand{\wuhao}{\fontsize{10.5pt}{\baselineskip}\selectfont}
\newcommand{\xiaowuhao}{\fontsize{9pt}{\baselineskip}\selectfont}
\newcommand{\liuhao}{\fontsize{7.875pt}{\baselineskip}\selectfont}
\newcommand{\qihao}{\fontsize{5.25pt}{\baselineskip}\selectfont}

\defaultfontfeatures{Mapping=tex-text}
\setmainfont{Droid Sans}% 设置缺省英文字体
\setCJKmainfont[BoldFont=SimHei,ItalicFont=KaiTi_GB2312]{SimSun}% 设置缺省中文字体

\renewcommand\refname{参考文献}

\newcommand{\upcite}[1]{\textsuperscript{\cite{#1}}} %自定义命令\upcite, 使参考文献引用以上标出现

\graphicspath{{figures/}}

\renewcommand\contentsname{目录}
\renewcommand\listfigurename{插图索引}
\renewcommand\listtablename{表格索引}
\newcommand\listequationname{公式索引}
\newcommand\equationname{公式}
\renewcommand\refname{参考文献}
\renewcommand\indexname{索引}
\renewcommand\figurename{图}
\renewcommand\tablename{表}
% \newcommand\CJKprechaptername{第}
% \newcommand\CJKchaptername{章}
% \newcommand\CJKthechapter{\@arabic\c@chapter}
% \renewcommand\chaptername{\CJKprechaptername~\CJKthechapter~\CJKchaptername}
\renewcommand\appendixname{附录}

\makeatletter
\newenvironment{tablehere}
  {\def\@captype{table}}
  {}

\newenvironment{figurehere}
  {\def\@captype{figure}}
  {}
\makeatother



\begin{document}
%  ABAP (R/2 4.3, R/2 5.0, R/3 3.1, R/3 4.6C, R/3 6.10)
% ACSL                               Ada (83, 95)
% Algol (60, 68)                     Assembler (x86masm)
% Basic (Visual)                     C (ANSI, Objective, Sharp)
% C++ (ANSI, GNU, ISO, Visual)       Caml (light, Objective)
% Clean                              Cobol (1974, 1985, ibm)
% Comal 80                           csh
% Delphi                             Eiffel
% Elan                               erlang
% Euphoria                           Fortran (77, 90, 95)
% GCL                                Haskell
% HTML                               IDL (empty, CORBA)
% Java (empty, AspectJ)              ksh
% Lisp (empty, Auto)                 Logo
% make (empty, gnu)                  Mathematica (1.0, 3.0)
% Matlab                             Mercury
% MetaPost                           Miranda
% Mizar                              ML
% Modula-2                           MuPAD
% NASTRAN                            Oberon-2
% OCL (decorative, OMG)              Octave
% Pascal (Borland6, Standard, XSC)   Perl
% PHP                                PL/I
% POV                                Prolog
% Python                             R
% Reduce                             Ruby
% S (empty, PLUS)                    SAS
% Scilab                             SHELXL
%                                    SQL
% Simula (67, CII, DEC, IBM)
% tcl (empty, tk)
% TeX (AlLaTeX, common, LaTeX, plain, primitive)
% VBScript                           Verilog
% VHDL (empty, AMS)                  VRML (97)
% XML
\lstset{
  language={[ANSI]C},
  linewidth=0.8\textwidth{},
  breaklines=true,
  numbers=left,
  numberstyle=\small{},
  stepnumber=5,
  numbersep=5pt,
  columns=fullflexible,
  basicstyle=\normalsize{}\ttfamily{},          % print whole listing small
  backgroundcolor=\color{black!10!white},
  frame=leftline,
  emph={czj,joyan},
  emphstyle=\color{blue}\bfseries,
  keywordstyle=\color{blue}\bfseries,
  identifierstyle=\color{black},           % nothing happens 
  escapeinside=`',
  commentstyle=\color{black}, % white comments
  stringstyle=\ttfamily,      % typewriter type for strings
  showstringspaces=false}     % no special string spaces

%%%%%%%%%%%%%%%%%%%%%%%%%%% 正文开始 %%%%%%%%%%%%%%%%%%%%%%%%%%%%%%


\title{一个简单的\LaTeX{}模板}
\author{陈志杰 \\ joyanlj@gmail.com}
\date{Feb. 2009}

\maketitle
  
\section{主要数据结构}

\appendix{}

\bibliography{bibdb}
\bibliographystyle{unsrt}


\end{document}
%%% Local Variables: 
%%% mode: latex
%%% TeX-master: t
%%% End: 
